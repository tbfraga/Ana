\documentclass[11pt]{book}
    \title{\textbf{Ana} \\ \vskip 1em \small The policy of silencing}
    \author{Tatiana Balbi Fraga}
    \date{}
    
    \addtolength{\topmargin}{-3cm}
    \addtolength{\textheight}{3cm}
\begin{document}

\maketitle
\thispagestyle{empty}

\chapter{about being alone}

I have always loved silence and tranquility. So being alone has never been a problem for me. Even isolated, I never felt really alone. \\

\noindent I once saw a report on TV5. The interviewee was a lady who was about 50 years old, with very short dark hair. This lady said she liked to go to a isolated place and be alone from time to time and that her family was unable to understand it. Trying to explain the feeling she sought she said: "Only when I'm alone I can find God." \\

\noindent I could completely understand this lady. I thought: That's right. \\

\noindent But, it was in recent years that I understood what it means to be alone... \\

\noindent \begin{center} \emph{Being alone means living the horror and having no one to ask for help.} \end{center}

\chapter{Sweetie}

\noindent It happened more than once. But the first time was something different. I was driving with a kitten on my lap. And submly I started to see the world with a different color. It is a subtle joy, without any foundation. A supreme well -being, a contently for nothing and anything. \\

\noindent The kitten was all black, very small. It was still a baby. In the corridors of universality. Someone passed and it followed the person. Then it came back to the same point, just to follow the next passer again. \\

\noindent There were many turns, until someone felt a strong interest in the kitten, and stopped to observe it with a certain curiosity. \\

\noindent I asked: Do you want to adopt it ? I promise that I give it to you castrated and dewormed. \\

\noindent Then I saw on her face a wide smile of comtemplation. \\

\noindent The kitten was so sweet that I named it Sweetie. I left Sweetie in my room at work until the end of the day. I tried to use a cardboard box to transport it to my house. But it ran away and while I was driving, Sweetie came to lodge in my lap. \\

\noindent I remember how happy I was, driving toward my house with that kitten. \\

\noindent In the following days, I cast it, and we spent a long time together until its complete recovery. Sweetie was very \emph {sapeca}. She was terribly happy at home. The days passed and I had to take Sweetie to its new owner. It was a little sad. Leaving Sweetie was very painful. It was like leaving a part of me. I donated the transportation box along with the kitten, and half a feed bag so that Sweetie could adapt well to its new home. \\

\noindent I said to my husband: I hope God takes good care of Sweetie. He replied: He is already taking care of. \\

\noindent Later, Sweetie recovered its old behavior. Someone passed and Sweetie followed the person, and returned again to the grocery store, just to follow the next person. Sweetie had to deal with another somewhat territorialist kitten. Its new mate did not want to share the love of its owners. \\

\noindent But Sweetie had its destination adjusted. And was adopted again. In Its new home, Sweetie found the company of a child. And happiness found both. Today Sweetie is a very loved and happy kitten. It's already adult, but it's still \emph{sapeca}. When its owner arrives home from college, Sweetie lies next to her. And keep looking at its owner typing on the computer. At night, Sweetie lodges on the feet of the bed. Look at its owner sleeping, gives a long supply and closes its eyes, waiting for the next day to arrive. \\

\chapter{The Covid Gang}

In spring, the heat came back in all its fullness. I decided I should put my concerns aside and go for a walk on the beach. After all, I left my apartment abandoned for a long time. I thought I deserved to take care of myself and that after a day or two on the beach I would be much more willing. \\

Friday was a perfect day. The trip to the beach was very fast and calm. Many people in the sand and many boats in the sea. All very different from what I remembered. But the day was beautiful. I walked all day in the beach sand and took a sea bath surrounded by fishes. I ate fish with capers sauce that was delicious, in an expensive restaurant . I bought a pot of acai that served to dinner. I read a little more of Laura Nowlin's book "If he was with me" while lying on the net. Everything was great. At night I slept like a stone. \\

On Saturday things change completely. Mr. Marcos da Fonte was an acquaintance. I found him casually early in the morning, while arguing with the doorman about the real need to have some kind of registration. I talked to Mr. da Fonte about condominium issues and told him about my kidnapping. \\

In the afternoon I'm reading on the net, and then suddenly I start to feel a terrible discomfort. I feel a very strong pain in the nostrils and a severe headache. Through the crack of the door, I see the cleaning lady passing cloth on the floor. \\

The malaise only increases and I decide to return home. I was thinking of leaving only after noon, but I advance my plans. Arriving at the post, I take a bottle of Gatorade at Freeser and ask for a cheese bread. I'm really feeling nauseous, as if I had drunk alcohol all night. It's been a long time since I've been drinking anything alcoholic, but the feeling is familiar to me. The boy takes my cheese bread through the door to warm it inside the kitchen. I think it's weird. I ask him why he didn't use the oven or the microwave in the canteen itself. He makes some not very reasonable excuse. In the middle of the trip, I enter a local avenue. I remember there was a supermarket I visited two weeks ago. As soon as I enter the avenue, a SAMU's vain comes out of somewhere and goes in front of me. I repeat to myself: just ignor it. And I follow my way. In the supermarket, as usual, there was an employee changing some cans, near the place where the mineral water bottles are. I see some crates of the bottle that I usually buy on the floor, next to this boy. On the shelf there were a few bottles like these. The bottles were covered with dust, as if they were there without care for a long time. Then I chose a different brand. I took some bottles that were not dusty. Almost arriving home, I stop in a good restaurant. I feed well and malaise passes the same time. While how, I see the attendant leave the counter a little stunned to talk to someone who seems to be the manager. I hear part of the conversation. The attendant says: she doesn't ... The alleged manager makes gestures and tells the attendant that he should move the food.\\

I get home and I feel relieved. Nothing better than being at home. I take a shower and take the book I was reading. After all, I really wanted to finish reading the book. I lie on the porch couch and start reading. \\

After a few hours of reading, some car at a reduced speed is on the road with the sound really very high. I stop reading a little and listen to the lyrics of the song that plays: "Take it, take it, take it in the ass." I have the impression that the car has stopped, because I hear the sound at the same height, without the sensation of which it is moving away. The chorus repeats himself a few times: "Take it, take it, take it in the ass." I try to ignore and go back to reading and then I realize that the volume of the sound slowly reduces until disappear. \\

The night is not very pleasant. I wake up feeling pain and discomfort again. It's still dawn. I hear some sound and decide to walk inside the condominium to find out where it comes from. I walk the condominium road following the music and find that the very high sound is installed in a narrow little deforested dirt road next to the condominium fence. I come home, I repeat to myself that my house is the safest place in the world and then I sleep again until dawn. \\

On Sunday, I take the day to write and hear a cyrene sound on the road several times throughout the day. It bothers me a little, but I try to ignore it.

\end{document}

