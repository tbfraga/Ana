\documentclass[11pt]{book}
    \title{\textbf{Ana} \\ \vskip 1em \small The policy of silencing}
    \author{Tatiana Balbi Fraga}
    \date{}
    
    \addtolength{\topmargin}{-3cm}
    \addtolength{\textheight}{3cm}
\begin{document}

\maketitle
\thispagestyle{empty}

\chapter{about being alone}

I have always loved silence and tranquility. So being alone has never been a problem for me. Even isolated, I never felt really alone. \\

\noindent I once saw a report on TV5. The interviewee was a lady who was about 50 years old, with very short dark hair. This lady said she liked to go to a isolated place and be alone from time to time and that her family was unable to understand it. Trying to explain the feeling she sought she said: "Only when I'm alone I can find God." \\

\noindent I could completely understand this lady. I thought: That's right. \\

\noindent But, it was in recent years that I understood what it means to be alone... \\

\noindent \begin{center} \emph{Being alone means living the horror and having no one to ask for help.} \end{center}

\chapter{Sweetie}

\noindent It happened more than once. But the first time was something different. I was driving with a kitten on my lap. And submly I started to see the world with a different color. It is a subtle joy, without any foundation. A supreme well -being, a contently for nothing and anything. \\

\noindent The kitten was all black, very small. It was still a baby. In the corridors of universality. Someone passed and it followed the person. Then it came back to the same point, just to follow the next passer again. \\

\noindent There were many turns, until someone felt a strong interest in the kitten, and stopped to observe it with a certain curiosity. \\

\noindent I asked: Do you want to adopt it ? I promise that I give it to you castrated and dewormed. \\

\noindent Then I saw on her face a wide smile of comtemplation. \\

\noindent The kitten was so sweet that I named it Sweetie. I left Sweetie in my room at work until the end of the day. I tried to use a cardboard box to transport it to my house. But it ran away and while I was driving, Sweetie came to lodge in my lap. \\

\noindent I remember how happy I was, driving toward my house with that kitten. \\

\noindent In the following days, I cast it, and we spent a long time together until its complete recovery. Sweetie was very \emph {sapeca}. She was terribly happy at home. The days passed and I had to take Sweetie to its new owner. It was a little sad. Leaving Sweetie was very painful. It was like leaving a part of me. I donated the transportation box along with the kitten, and half a feed bag so that Sweetie could adapt well to its new home. \\

\noindent I said to my husband: I hope God takes good care of Sweetie. He replied: He is already taking care of. \\

\noindent Later, Sweetie recovered its old behavior. Someone passed and Sweetie followed the person, and returned again to the grocery store, just to follow the next person. Sweetie had to deal with another somewhat territorialist kitten. Its new mate did not want to share the love of its owners. \\

\noindent But Sweetie had its destination adjusted. And was adopted again. In Its new home, Sweetie found the company of a child. And happiness found both. Today Sweetie is a very loved and happy kitten. It's already adult, but it's still \emph{sapeca}. When its owner arrives home from college, Sweetie lies next to her. And keep looking at its owner typing on the computer. At night, Sweetie lodges on the feet of the bed. Look at its owner sleeping, gives a long supply and closes its eyes, waiting for the next day to arrive. \\

\end{document}

