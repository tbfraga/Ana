\documentclass[11pt]{book}
    \title{\textbf{Ana} \\ \vskip 1em \small The policy of silencing}
    \author{Tatiana Balbi Fraga}
    \date{}
    
    \addtolength{\topmargin}{-3cm}
    \addtolength{\textheight}{3cm}
\begin{document}

\maketitle

\chapter*{Abstract}

'Ana: The Policy of Silencing' is a fiction book that discusses some important issues of our century, such as: COVID; technological advancements; the loss of privacy and individuality; immediacy; competition; the industry of madness; posthumanism; transhumanism, etc. \\

\noindent The book uses an adaptation of the textual language used by Laura Nowlin's in her book 'If He Was With Me'. Nowlin narrates the past using present and future time. It's interesting. \\

\noindent However, the idea of the book is to design a web of memories and reflection, creating a story that walks through a brain's synaptic networks... \\

\noindent It is not possible to know if it will be good until it is over. \\

\noindent And there is no intention to finish it. \\

\noindent In short ... it's just something for distraction ... but also a provocation in the hope that things will change at some point. \\

\thispagestyle{empty}

\chapter{about being alone}

I have always loved silence and tranquility. So being alone has never been a problem for me. Even isolated, I never felt really alone. \\

\noindent I once saw a report on TV5. The interviewee was a lady who was about 50 years old, with very short dark hair. This lady said she liked to go to a isolated place and be alone from time to time and that her family was unable to understand it. Trying to explain the feeling she sought, she said:

\noindent \begin{center} "Only when I'm alone I can find God." \end{center}

\noindent I could completely understand this lady. I thought: That's right. But, it was in recent years that I understood what it means to be alone...

\noindent \begin{center} \emph{Being alone means living the horror and having no one to ask for help.} \end{center}

\chapter{Sweetie}

\noindent It happened more than once. But the first time was something different. I was driving with a kitten on my lap. And submly I started to see the world with a different color. It is a subtle joy, without any foundation. A supreme well -being, a contently for nothing and anything. \\

\noindent The kitten was all black, very small. It was still a baby. In the corridors of universality. Someone passed and it followed the person. Then it came back to the same point, just to follow the next passer again. \\

\noindent There were many turns, until someone felt a strong interest in the kitten, and stopped to observe it with a certain curiosity. \\

\noindent I asked: Do you want to adopt it ? I promise that I give it to you castrated and dewormed. \\

\noindent Then I saw on her face a wide smile of comtemplation. \\

\noindent The kitten is so sweet that I named it Sweetie. I left Sweetie in my room at work until the end of the day. I tried to use a cardboard box to transport it to my house. But it ran away and while I was driving, Sweetie came to lodge in my lap. \\

\noindent I remember how happy I was, driving toward my house with that kitten. \\

\noindent In the following days, I cast it, and we spent a long time together until its complete recovery. Sweetie was very \emph {sapeca}. She was terribly happy at home. \\

\noindent The days passed and I had to take Sweetie to its new owner. It was a little sad. Leaving Sweetie was very painful. It was like leaving a part of me. I donated the transportation box along with the kitten, and half a feed bag so that Sweetie could adapt well to its new home. \\

\noindent I said to my husband, Adan: I hope God takes good care of Sweetie. He replied: He is already taking care of. \\

\noindent Later, Sweetie recovered its old behavior. Someone passed and Sweetie followed the person, and returned again to the grocery store, just to follow the next person. Sweetie had to deal with another somewhat territorialist kitten. Its new mate did not want to share the love of its owners. \\

\noindent But Sweetie had its destination adjusted. And was adopted again. In Its new home, Sweetie found the company of a child. And happiness found both. Today Sweetie is a very loved and happy kitten. It's already adult, but it's still \emph{sapeca}. When its owner arrives home from college, Sweetie lies next to her. And keep looking at its owner typing on the computer. At night, Sweetie lodges on the feet of the bed. Look at its owner sleeping, gives a long supply and closes its eyes, waiting for the next day to arrive. \\

\chapter{The Covid Gang}

\noindent In spring, the heat came back in all its fullness. I decided I should put my concerns aside and go for a walk on the beach. After all, I left my apartment abandoned for a long time. I thought I deserved to take care of myself and that after a day or two on the beach I would be much more willing. \\

\noindent Friday was a perfect day. The trip to the beach was very fast and calm. Many people in the sand and many boats in the sea. All very different from what I remembered. But the day was beautiful. I walked all day in the beach sand and took a sea bath surrounded by fishes. I ate fish with capers sauce that was delicious, in an expensive restaurant . I bought a pot of acai that served to dinner. I read a little more of Laura Nowlin's book "If he was with me" while lying on the net. Everything was great. At night I slept like a stone. \\

\noindent On Saturday things change completely. Mr. Marcos da Fonte was an acquaintance. I found him casually early in the morning, while arguing with the doorman about the real need to have some kind of registration. I talked to Mr. da Fonte about condominium issues and told him about my kidnapping. \\

\noindent In the afternoon I'm reading on the net, and then suddenly I start to feel a terrible discomfort. I feel a very strong pain in the nostrils and a severe headache. Through the crack of the door, I see the cleaning lady passing cloth on the floor. \\

\noindent The malaise only increases and I decide to return home. I was thinking of leaving only after noon, but I advance my plans. Arriving at the post, I take a bottle of Gatorade at Freeser and ask for a cheese bread. I'm really feeling nauseous, as if I had drunk alcohol all night. It's been a long time since I've been drinking anything alcoholic, but the feeling is familiar to me. The boy takes my cheese bread through the door to warm it inside the kitchen. I think it's weird. I ask him why he didn't use the oven or the microwave in the canteen itself. He makes some not very reasonable excuse. In the middle of the trip, I enter a local avenue. I remember there was a supermarket I visited two weeks ago. As soon as I enter the avenue, a SAMU's vain comes out of somewhere and goes in front of me. I repeat to myself: just ignor it. And I follow my way. In the supermarket, as usual, there was an employee changing some cans, near the place where the mineral water bottles are. I see some crates of the bottle that I usually buy on the floor, next to this boy. On the shelf there were a few bottles like these. The bottles were covered with dust, as if they were there without care for a long time. Then I chose a different brand. I took some bottles that were not dusty. Almost arriving home, I stop in a good restaurant. I feed well and malaise passes the same time. While how, I see the attendant leave the counter a little stunned to talk to someone who seems to be the manager. I hear part of the conversation. The attendant says: she doesn't ... The alleged manager makes gestures and tells the attendant that he should move the food.\\

\noindent I get home and I feel relieved. Nothing better than being at home. I take a shower and take the book I was reading. After all, I really wanted to finish reading the book. I lie on the porch couch and start reading. \\

\noindent After a few hours of reading, some car at a reduced speed is on the road with the sound really very high. I stop reading a little and listen to the lyrics of the song that plays: "Take it, take it, take it in the ass." I have the impression that the car has stopped, because I hear the sound at the same height, without the sensation of which it is moving away. The chorus repeats himself a few times: "Take it, take it, take it in the ass." I try to ignore and go back to reading and then I realize that the volume of the sound slowly reduces until disappear. \\

\noindent The night is not very pleasant. I wake up feeling pain and discomfort again. It's still dawn. I hear some sound and decide to walk inside the condominium to find out where it comes from. I walk the condominium road following the music and find that the very high sound is installed in a narrow little deforested dirt road next to the condominium fence. I come home, I repeat to myself that my house is the safest place in the world and then I sleep again until morning. \\

\noindent On Sunday, I take the day to write and hear a cyrene sound on the road several times throughout the day. It bothers me a little, but I try to ignore it.

\chapter{Adam}

\noindent We are walking in the beach sand in Mikonos. We see a great rock a little distant from the coast, as if it were a small rocky island being snatched by the sea. Adam propose that we swim to the island, and of course ... I would never refuse. \\

\noindent Sea water is warm and transparent. We are wearing diving glasses and we swim to the island, submerged and hand in hand ... \\

\noindent The island is a little larger than I expected. We explore the island and, like two children, we try to hide when we see a boat coming close to us. \\

\noindent On the way back, I dive into the sea with my hands forward, as if I were a dolphin. But I do this wearing the diving glasses that break when shocking with the water. I swim to the coast with my forehead bleeding, but with a big smile on my face ... \\

\noindent Weeks later, we are already back in France ... I walk next to Adam ... \\

\noindent Adam tells me about a recent dream. He tells me that we are walking hand in hand on the sea, as if we were walking on a thin, transparent layer of ice. And that, under our feet, we can see thousands of fish. I can see Adam's dream like a memory. \\

\noindent I have no doubt about the meaning of the dream. It is just a reflection of the feeling we have for each other.

\chapter{Day before my kidnapping}

\noindent After driving many hours, I arrive at my beach apartment accompanied by my mother and my cousin, Basco. I am really exhausted and I feel that I will have difficulty giving the medication to my kittens. I let out the kittens in the bedroom and, with a lot of effort, I put the sandbox, feed and water for them and I can medicate Braquinha and Lucy. Then I fall on the bed and finally sleep like a stone. \\

\noindent I wake up after several hours, feeling very hot. I'm still really tired. The deep sleep I had was not enough to recover my disposition. The room is muffled, in a way that cannot be supported. I open the bedroom door and I free my kittens in the apartment.  \\

\noindent I walk to the living room and see my mother lying on the bicama couch. She has a somewhat strange expression, as if she was feeling terribly terrible with something. Basco is not in the apartment. The kitchen sink drain is covered with a water-filled cuzcuzeira, and the bathroom sink drain is clogged with colored toothpaste. The sink is full of water. Given what I've been through in recent days, I didn't find it strange. But I said to my mother: Mom, the kittens are drinking water from the bathroom sink ... this will make them sick.  \\

\noindent My mother goes to the bathroom, empties and cleans the sink. \\

\noindent Hours later, several people arrive at the apartment door. My mother opens the door and a slightly strong gentleman with a somewhat rounded face crosses the door. My mother leaves the door open as Mr. Rounded Face begins to question me, without worried about a possible glow of my kittens. A condominium employee awaits outside, accompanied by unknown persons. I am worried about the door open and my kittens, but the lord with rounded face begins to question me abruptly. \\

\noindent I tell this gentleman about why I want my mother to leave. I answer the questions narrating a long story. Although I am confused with what was happening, I came to some kind of conclusion. \\

\noindent When he finishes his interrogation, Mr. Rounded Face turns to my mother and says: we can't take her like this, she's fine. \\

\noindent The people who were outside move away. Mr. Rounded Face begins a somewhat strange speech. He proposes that I walk on the beach with my mother so someone he has entitled "she" couldn't hear us. My mother has an espression that becomes more restless as the Lord's speech continues. Then he closes the door of the apartment and says: Since your mother is here, I will say: Don't exceed yourself because we will make you suspect everyone.  \\

\noindent I don't know what to think about all this. Nothing seems to have some kind of coherence, but I try to put the facts together. \\

\noindent My mother is washing dishes and I stop by her side and ask: is you the person responsible for all this ? \\

\noindent My mother bends her body down forming on her back a hunchback. She tilts her head to the side and down. Then she shakes her head positively, speaking softly: Yes, it's me. \\

\noindent I tell my mother: I want you to leave. \\

\noindent And my mother resists with a certain insistence. \\

\noindent After a while, my mother gives me the phone, saying that I should talk to my sister. \\

\noindent I say on the phone: Thaila, I want mom to leave. And Thaila replies: Ana, you have a niece. Confused, I return the phone to my mother and say: You leave. \\

\noindent After a few minutes my mother comes to talk to me. She says: I'm leaving tomorrow, but only if you talk to a doctor. He will call at 19:00h. \\

\noindent We await the call, the doctor does not call at 19:00h and he does not call later. I tell my mother: Well, no doctor called. And my mother replies: he may have suffered an accident. Again confused, I say: I already waited enough, I'm tired, I'm going to sleep. Tomorrow you leave. \\

\chapter{Genius and madness}

\noindent Many people believe that genius is a gift that arises when geniuses are born. But this is not totally true since every being is the result not only of genetics, but also of a set of experiences. \\

\noindent I believe that:

\noindent \begin{center} \emph{Genius happens after total abdication of itself through self-delivery to a passion for which the genius being has aptitude} \end{center} 

\noindent In history, many geniuses are portrayed as crazy at some point in their lives, as exemple:

\begin{itemize}
\item During his doctorate, and before redefining Game Theory, John Nash talked to invisible people; 
\item In the last years of his career, Albert Einstein used to go to work using pajamas. 
\end{itemize}

\noindent I will not comment here that, in 1994, John Nash was awarded the Nobel Memorial Prize in the Economic Sciences for his contribution to game theory. \\

\noindent About Einstein, I feel as if I knew him deeply and I can guarantee that going to work wearing pajamas was for Einstein a form of mockery to society's standards. \\

\noindent For a long time, Einstein's brilliance was deliberately hidden due to the envy of co-workers. After many years he was finally recognized and became a key element for the university's image he worked for. Einstein had numerous scientific publications and discoveries that changed a generation. Therefore, when Einstein understood that his place in the world was already sealed, he became a better version of himself, abandoning behaviors that were now dispensable for him. \\

\noindent The story of these crazy geniuses brings out an important question: what is madness ? \\

\noindent And, based on my recent experiences, I can present a possible answer:

\noindent \begin{center} \emph{Madness is a behavior that is not accepted by an elite of the social group in which the madman is inserted.} \end{center}

\noindent From the madman begins to emerge new ideas and questions related to standards, ethical questions, principles and truths. Concepts which such an elite does not want to dig up.

\noindent \begin{center} \emph{Madness is therefore a way to silence the provocations of a genius} \end{center}

After all, attesting that a person is mad, it is a very effective way to discredit them.  \\

\noindent One thing is certain about the geniuses:

\noindent \begin{center} \emph{A genius sees things that other people can't see, in places where other people don't expect} \end{center}

\noindent So it is natural for genius people not to be understood. And the fact that geniuses are always linked to change, it can really scare. \\

\noindent Therefore, people's cruelty is not associated with fear or lack of understanding.

\noindent \begin{center} \emph{People's cruelty is linked to the cruel, selfish and competitive nature, inherent in the human being.} \end{center}

\noindent And madness is clearly a concept created by cruel people.

\chapter {COVID Gang - Who are they ?}

\noindent When you think of COVID, you cannot see this in a purely personal aspect ... But in fact, COVID has become personal for thousands, millions or even billions of people. COVID didn't just happen to me. But it happened to me lately in a way as inhuman and frightening as torture and terrorism can be. \\

\noindent Thinking about the personal aspect, the COVID Gang can be distributed in two classes: people who are financing the aggressions; and people who are performing the aggressions. \\

\noindent Who are the people who are financing this ? \\

\noindent I paid an appointment to two lawyers. I told them about the last torments I have had to endure. \\

\noindent We are in our third meeting and the lawyers have a budget in their hands. The lawyer tells me: I believe that who is doing this is a group of people. \\

\noindent The lawyer is right, but who are these people ? \\

\noindent In my opinion, if it is really a group, this group could be formed by: owners of a psychiatric clinic, employees of a technology company, people from my work; other public officials; a corrupt lawyer I hired in the past; people from my beach condominium; the construction company that sold me the beach apartment; person who sold me the land in which I built my house. Most of these people, assaulted me strongly in very different ways, long before COVID begins. These are the people who might want to silence me or make me look crazy and disappear, buried alive in a clinic.\\

\noindent And who are the people who are directly terrifying me ? \\

\noindent Criminals and ordinary people who are unemployed or earn a low salary such as cashier attendants, truckers, restaurant employees, gas stations attendants. \\

\noindent It is possible to understand why the "hate financiers" are assaulting me. \\

\noindent But why do ordinary people are assaulting me ? \\

\noindent There are many possible reasons, but I have some guesses that seem to present some kind of logic:

\begin {itemize}
\item some people have been hired in some way for this;
\item someone created some kind of hatred chain against me;
\item criminals are assaulting other people and making it seens that I am responsible for this;
\item criminals are assaulting or threatening other people who refuse to assault me;
\item Some people just feel good about doing this.
\end{itemize}

\noindent Possibly all these alternatives are correct. The only thing I can think is that somehow people were convinced to cooperate in this regard. \\

\noindent This is the logical conclusion that I arrived after almost two years being tortured and terrified. But ... is there any other explanation behind it ?

\chapter{A good opportunity}

\noindent It's been over a year since I started my doctorate. Nancy is an excellent workmate. It's been a few years since we started working in the same room. We divide our dawn in front of our computers ... reading ... typing ... searching. We have different goals for a common end: research. And it seems clear, we love what we do. \\

\noindent We talk little, but our short conversations are always pleasant. We talk about our lives, the longing for a daughter, the need to pay bills from a distant country ... We live in a small town, but we never talk bad about other people. \\

\noindent Nancy always gives me some important tips that I will lead to the rest of my life, such as about CodeBlocks, cplusplus.com, or about some technical concept or even how to overcome some buracratic challenges of the university. \\

\noindent We are on a cold night squeezing our arms under our already aged wool coats. Nancy tells me about doctoral scholarships abroad that are offered by Capes and CNPq ... \\

\noindent I always wanted to know Europe. Especially Greece. I saw a few Greek movies, and I was always dazzled by the scenes that happened in the Greek rocks. Always an amazing landscape. And ... well ... the gods are Greek. \\

\noindent For this reason, I spent the next few days trying to understand a vast paperwork of edicts. \\

\noindent Then, I visit my advisor's room with a simple task: teacher, there is a doctoral internship bag abroad, I need you to present me a reference from someone you have worked with. \\

\noindent And he answered me: there is a teacher in the USA, another in Portugal, and finally, Professor Bernardo, who works at a university in France. \\

\noindent USA... no, no, and no ... Portugal ... all there they speak Portuguese ... \\

\noindent The next days were nothing easy ... \\

\noindent I'm crying in front of the computer because I don't understand the meaning of \emph{on}, \emph{y}: \emph{On y vas ? On a parlé.}  \\

\noindent I have little time. I need to take a grade 7.0 in the French Alliance test. Toffel-equivalent examination for the french language. \\

\noindent I paid a few classes to a French teacher. She gave me some important tips on TV5 and a French radio. But after three meetings, she told me very sincerely: you won't make it. \\

\noindent At this point I introduce something very important about my personality: In my life, I never measure efforts. \\

\noindent I create a learning technique totally outside any standard of reality. \\

\noindent I chose some videos from the channel apprende TV, from the TV5 website, so I decide to listen to the videos, then read and translate the transcription, then listen again, then listen and write. I make it repeatedly, until I could understand speech and write the  text 100\% in french without any grammatical error. Then I start doing the same with french radio audios nominated by the teacher. \\

\noindent After three months repeating this repertoire, I perform the French Alliance test. And my grade is: 6.5. \\

\noindent I'm a little unhappy, because I just have one more chance.  \\

\noindent I make another appointment with the French teacher. After all, the tips she gave me helped me a lot. \\

\noindent After a few minutes of conversation in french, the teacher says: I'm surprised. How did you get this ? \\

\noindent I answer: following your tips, and making an absurd effort. \\

\noindent After another month, I perform the French Alliance test again. And my grade is: 7.5. \\

\noindent And at this point I introduce other very important thing about me: I could never ever correctly pronounce the word \emph{apprendre}. Can anyone do this ?

\chapter{COVID}

\noindent I'm working in my bedroom. Adam calls me with a certain enthusiasm. I go to the TV room. Adam shows me a video of something that happens in China. People are leaving some city. The temperature of each person who goes through some kind of portal is measured with some instrument that seems to have the shape of a bar code reader. \\

\noindent I try to understand the English report with a certain effort. I understand part of the speech. The scene I see seems to repeat some movie about the Holocaust. \\

\noindent Adam tells me: How is it possible that they are letting people get out of town ? \\

\noindent I go back to my room and I Google the word sars cov virus. \\

\noindent I read some reports about Asian influenza and I try to understand as deeply as possible the type of viruses being addressed in the report. \\

\noindent I go back to the TV room, and I keep looking at the report next to Adam, now completely stagnant.

\chapter{Mr. Rocher}

\noindent I wake up at dawn feeling a severe headache. I swallow a pain medicine along with some sips of water. I turn on my computer and log into the game to do some missions hoping the pain pass. I'm really sleepy. I want to go back to sleep, but the pain bothers me. \\

\noindent Mr. Rocher soon finds me online and asks: Are you awake ? I answer: I have a headache, I can't sleep. \\

\noindent We talk about something of the game. There will soon be a battle between servers. Mr. Rocher is one of the first covenant ministers. He is responsible for diplomatic matters and is always solving disputes and seeking agreements. However, as one of the first ministers, he is also always concerned with battles. \\

\noindent I just adore him. While we talk I feel calm, safe and happy. \\

\noindent It's been about two years since we met. Our wedding happened a few months after someone decided to turn the game into a relationship site. Several wedding packages were launched with real prices varying according to the size of the party. We got the cheaper package, but we had a typical Russian wedding, with a lot of greetings and long verses with good fortune and a lasting relationship desires. It was a very funny, cheerful and even a little exciting moment. It was my first marriage. \\

\noindent After long minutes of a very pleasant conversation, we say goodbye long and then, without any trace of headache and with a wide smile on my face, I can go back to sleep.

\end{document}

