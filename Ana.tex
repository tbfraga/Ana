\documentclass[11pt]{book}
    \title{\textbf{Ana} \\ \vskip 1em \small The policy of silencing}
    \author{Tatiana Balbi Fraga}
    \date{}
    
    \addtolength{\topmargin}{-3cm}
    \addtolength{\textheight}{3cm}
\begin{document}

\maketitle
\thispagestyle{empty}

\chapter{}

I have always loved silence and tranquility. So being alone has never been a problem for me. Even isolated, I never felt really alone. \\

\noindent I once saw a report on TV5. The interviewee was a lady who was about 50 years old, with very short dark hair. This lady said she liked to go to a isolated place and be alone from time to time and that her family was unable to understand it. Trying to explain the feeling she sought she said, "Only when I'm alone I can find God." \\

\noindent I could completely understand this lady. I thought: That's right. \\

\noindent But, it was in recent years that I understood what it means to be alone... \\

\noindent \begin{center} \emph{Being alone means living the horror and having no one to ask for help.} \end{center}





\end{document}

