\documentclass[11pt]{book}
    \title{\textbf{Ana} \\ \vskip 1em \small The policy of silencing}
    \author{Tatiana Balbi Fraga}
    \date{}
    
    \addtolength{\topmargin}{-3cm}
    \addtolength{\textheight}{3cm}
    

 \usepackage[T1]{fontenc}
 \usepackage[utf8]{inputenc}
% \usepackage{amsmath,amssymb}
 %\usepackage[T2A]{fontenc}
 %\usepackage[russian,english]{babel}
 \usepackage{graphicx}
 
\begin{document}

\maketitle

Eu fiz um estágio de doutorado na França, na cidade de Rouen. Foi quando eu conheci Abdeladhim (Adim), meu ex companheiro. Depois que terminei meu doutorado, eu voltei para a França porque Adim me pediu em casamento.

Contudo, a Europa havia entrado em um crise financeira, por isso eu tive dificuldades de conseguir um pós-doc na França e eu vim para Pernambuco aproveitando um oportunidade de pós-doutorado no departamento de engenharia mecânica da UFPE, em Recife. Eu fui orientada pelo professor Paulo Lyra e pelo professor Urtiga. Eu estava trabalhando com métodos de diferenças finitas para simulação de produção de petróleo. Eu aprendi muito, mas depois de dois anos eu entendi que eu precisava de publicações, e que eu levaria muito tempo para conseguir publicar nesta área. 

Então eu procurei o departamento de engenharia de produção e me informaram que havia um concurso para Caruaru. Eu estudei bastante e consegui passar no concurso. Mas logo no início, eu tive um problema com o grupo de engenharia de produção. O professor Adiel (da engenharia de produção da UFPE, campus de Recife) me chamou para conversar sobre um projeto, mas quando eu fui encontrá-lo, ele apenas me criticou, informando que as pessoas do grupo de engenharia de produção estavam apresentando várias críticas sobre mim. A principal questão foi relacionada a um comentário que eu fiz sobre o filho dele, que havia passado no mesmo concurso que eu passei, mas para uma outra área. Eu comentei brincando que várias vezes o professor Adiel repetia para mim que havia uma vaga de concurso e depois se lembrava que já havia uma pessoa para a vaga. Então quando descobri que o filho dele havia prestado o mesmo concurso, eu disse brincando que entendi o comportamento do professor Adiel. 

Devido as diversas críticas que recebi na conversa que eu tive com o professor Adiel, eu sai da universidade chorando. 

Depois disso, eu decidi me afastar do grupo de professores de engenharia de produção. Eu montei o meu próprio grupo de pesquisa com professores de outros núcleos ou departamentos e desenvolvi meu trabalho de forma independente.

Depois de alguns anos, eu estava conversando sobre uma disciplina que eu lecionava e mostrei para o coordenador do curso de engenharia de produção do Centro Acadêmico do Agreste, na época professor Lúcio, o trabalho que eu estava fazendo na disciplina e as publicações que eu estava tendo com os alunos. Depois disso, começou a se desenrolar um forte boicote ao desenvolvimento de meu trabalho, que se agravou como uma bola de neve, e resultou em uma série de agressões. Um pouco antes do surgimento do COVID, conseguiram destruir a minha imagem junto à pró-reitoria de pesquisa, informando que eu estava roubando os trabalhos desenvolvidos por meus alunos. Por conta disso, eu não conseguia registrar mais nenhum projeto de pesquisa. Ação que é necessária para o desenvolvimento de pesquisa dentro da UFPE.

Chegaram a falsificar um processo físico meu, na tentativa de arquivar um de meus relatórios, eliminando qualquer recursos que eu tentasse utilizar para me defender buscando aprovar o registro de meus projetos de pesquisa.

Eu fiquei muito deprimida, imaginando que não havia nada que eu pudesse fazer para sair deste cíclo, mas então veio o COVID e eu tive a oportunidade de me afastar, o que me permitiu uma recuperação psicológica. 

Eu consegui submeter um livro para publicação, participei do primeiro processo seletivo do desenvolve.aí e tive o meu projeto selecionado como prioridade número 1 da empresa BALL Corporation. Contudo, quando eu voltei às atividades, os problemas retornaram. Após selecionado, meu projeto do desenvolve.aí foi passado para o CESAR, isso após eu ter agradecido pela ajuda de algumas pessoas no grupo de professores. Também não estava conseguindo publicar meu livro que estava na fase de diagramação pela editora Apris. Me enviavam a versão para aprovação com erros. Eu apresentava as correções para a editora e voltava uma nova verção com novos erros. Isso aconteceu inúmeras vezes.

Meu colega de trabalho, Marcos Henrique, me acompanhou em uma reunião com a diretoria de pesquisa da UFPE. Após a reunião ele me disse: Tati, fizeram a sua caveira. Nada que você faça agora vai te ajudar, então a gente precisa publicar.

Por conta desta conversa que eu tive com o professor Marcos, depois que o projeto que eu desenvolvi no processo de seleção desenvolve.aí foi passado para o CESAR, eu decidi terminar um dos trabalhos que eu desenvolvi com meus orientandos para publicação.

Eu estava trabalhando na elaboração de um artigo para este trabalho de orientação e na revisão da diagramação de meu livro no final do ano de 2021. Quando minha mãe me ligou e disse que queria vir ficar no meu apartamento de praia em Carneiros. Eu disse para minha mãe que não poderia dar atenção a ela naquele momento. Mas minha mãe me disse que minha avó ficava gritando durante a noite e que ela precisava de um descanso.

Então disse que minha mãe poderia vir, mas que eu não poderia das atenção a ela. Enquanto minha mãe esteve em Carneiros, eu pude visitá-la apenas duas vezes. Mas conversávamos pelo telefone. 

Durante este período, houve uma conversa no grupo de professores e eu acabei reclamando de todas as dificuldades que eu estava enfrentando. Eu falei sobre o fato de não estar conseguindo publicar meu livro e também sobre o fato que meu projeto do desenvolve.aí foi passado para o CESAR, entre outros problemas.

Eu já estava tendo problemas com gente esvaziando os pneus do meu carro e abrindo buracos na tela do gatil, em minha residência. Além disso, eu já estava tendo problemas com o fornecimento de internet e de energia elétrica. Mas depois que eu reclamei no grupo de professores sobre os problemas que ocorreram no ambiente de trabalho, as agressões começaram a se agravar. Em dezembro de 2021, quando minha mãe já havia retornado para o estado do Rio de Janeiro, começaram e me causar dor. Isso ocorreu logo depois que arrancaram uma pedra em baixo da janela do banheiro de minha casa. Como se o desnível criado fosse utilizado para apoio dos pés.

Acredito que, inicialmente, a dor foi causada inserindo algum produto tóxico pela janela do banheiro, da suite onde eu durmo. 

Diversas outras formas de agressão foram iniciadas. Em Janeiro de 2022, liguei para minha mãe e ela me disse que eu deveria pegar o carro e ir para o estado do Rio de Janeiro, ficar junto de minha família. Eu não tinha como levar os meus gatinhos de avião. Então minha mãe sugeriu que a viagem fosse feita de carro. 

Eu estava de férias. Eu fui então de carro com meus gatinhos de Chã Grande até Sergipe. Parando várias vezes. Mas eu fui aterrorizada em toda a viagem. Quando eu cheguei em Sergipe, eu já estava passando muito mal. E além disso o carro parou de funcionar. Ele morreu no meio da estrada e eu não estava conseguindo ligar o motor. Eu consegui ligar para um oficina. Eu fui rebocada aré a oficina. Depois meu ex-companheiro Adim veio me ajudar e me levou junto com meus gatinhos para um apartamento que minha irmã havia reservado em Sergipe. Fomos para o apartamento em Sergipe, em um carro alugado. No caminho, adim sugeriu que eu tentasse pegar um avião para o Rio de Janeiro, por isso fomos para o areoporto de Sergipe. Ficamos aguardando horas no aeroporto. Eu não sei informar por qual motivo. 

Chegando no apartamento de Sergipe, meu ex companheiro, Adim, agiu de forma estranha. Por exemplo, ele perguntou ao atendente de uma loja se ele poderia comprar água, recebendo a resposta de que poderia levar apenas uma garrafa. Meu ex companheiro também ligou o ventilador, deixando ventilar várias bolinhas de isopôr. Enquanto eu estava dormindo no apartamento, eu acordei sentindo muita dor. Depois que eu discuti com meu ex companheiro, ele disse que havia informado a minha mãe que não iria mais participar do que estava acontecendo, e que minha mãe viria ao meu encontro.

Minha mãe veio junto com meu primo para Sergipe. Para que continuássemos a viagem de carro até o Rio de Janeiro. Contudo, eu insisti que a viagem era absurda e que eu queira retornar para casa. 

Minha mãe alugou um carro. Fomos até a oficina em Sergipe pegar o meu carro, que ficou na oficina para ser concertado. Eu voltei para Carneiros dirigindo o meu carro, e meu primo dirigiu o carro alugado por minha mãe. Eu estava ainda assustada, mas fiquei feliz porque meus gatinhos estavam bem e eu estava voltando para casa com minha mãe.

No caminho, de Sergipe à Carneiros, minha mãe me informou que disseram a ela que ou eu iria para uma clínica ou para o SAMU. No entanto minha mãe se negou a dizer quem lhe disso isso.

Quando chegamos em Carneiros, eu queria deixar minha mãe e primo no meu apartamento da praia e voltar para minha casa em Chã Grande com os gatinhos. Contudo, eu estava muito cansada e precisava dormir. Enquanto eu fui dormir no quarto, minha mãe enxeu as pias da cozinha e do banheiro com água. Ela tampou o ralo da pia da cozinha com uma panela cheia de água e preencheu o ralo da pia do banheiro e toda a pia com pasta de dente colorida. Ela estava agindo de forma muito estranha. Suas expressões faciais eram muito estranhas. Quando acordei, veio um senhor em meu apartamento me interrogar. Ele estava acompanhado por meu primo e funcionários do condomínio que permaneceram do lado de fora da porta. Depois me disseram que era alguém do SAMU. Após me interrogar esta pessoa disse para minha mãe: não é possível levá-la, pois ela esta bem. Contudo, quando meu primo e funcionário do condomínio saíram, ele fechou a porta e disse que eu deveria ficar tranquila que eles iriam me fazer desconfiar de todo mundo. 

No dia seguinte, quando minha mãe estava indo embora, eu fui abordada por 3 pessoas com uniforme do exército, que estavam conversando com a minha mãe, no térreo do bloco onde fica meu apartamento de Carneiros, dentro do condomínio. Estas pessoas pediram para que eu as acompanhasse e depois me levaram para a clínica Novo Nascer em Camaragibe. Quando o carro chegou na portaria, saindo do condomínio, uma quarta pessoa, também vestida com uniforme do exército pegou a minha bolsa à força. Eu gritei dentro do carro, mas nenhum segurança do condomínio veio me defender. Depois devolveram minha bolsa sem as minhas chaves, esta quarta pessoa entrou dentro do carro, e me levaram para a clínica Novo Nascer, em Camaragibe. Depois que cheguei na clínica, uma das pessoas entrou na clínica, chamando alguém para me acompanhar. E então fui levada para conversar com uma médica. Sobre o que ocorreu em seguida, eu já informei. 


\end{document}

